%% vitamin D$_3$ and survival from Kidney cancer (K2 study)
%% tables and figures

\documentclass[a4paper,10pt]{article}
%\usepackage[numbers,comma,square,sort&compress]{natbib}
\usepackage{cite}
\renewcommand{\citeleft}{(}
\renewcommand{\citeright}{)}
\usepackage{setspace}

\usepackage{caption}

\usepackage[left=2.5cm,right=2.5cm,top=2.5cm,bottom=2.5cm]{geometry}
\usepackage{graphicx}
\usepackage{amsmath}
\usepackage{amssymb}


\usepackage{threeparttable}
\usepackage{color, soul}
\usepackage{booktabs}
\usepackage{multicol}
\usepackage{pdflscape}
\usepackage{hyperref}
\hypersetup{
  colorlinks=true,
  urlcolor=black,
  citecolor=black,
  linkcolor=black,
}

%% use palatino font
\usepackage{mathpazo}
\linespread{1.15}

%% indenting and paragraphing
\parskip 10pt
\parindent 25pt

%\renewcommand{\cite}{\citep}

\begin{document}

%% descriptive table
\begin{table}
\caption{Demographic and clinical characteristics of the participants by vital 
status at the end of follow-up.}
\centering
\begin{tabular}{llccccc}
\toprule
& & \multicolumn{4}{c}{Vital status} & \multicolumn{1}{c}{} \\ \cmidrule(lr){3-6}
& & \multicolumn{2}{c}{alive} & \multicolumn{2}{c}{dead} & \multicolumn{1}{c}{} \\ \cmidrule(lr){3-4}\cmidrule(lr){5-6}
 &  & n & (\%) & n & (\%) & \multicolumn{1}{c}{Total} \\ 
\midrule
 & Total  & $427$ & \multicolumn{1}{@{\hspace{\tabcolsep}(}r@{)\hspace{\tabcolsep}}}{$100$} & $203$ & \multicolumn{1}{@{\hspace{\tabcolsep}(}r@{)\hspace{\tabcolsep}}}{$100$} & $630$ \\
 & \\ %  &  & \multicolumn{1}{@{\hspace{\tabcolsep}(}r@{)\hspace{\tabcolsep}}}{$100$} &  & \multicolumn{1}{@{\hspace{\tabcolsep}(}r@{)\hspace{\tabcolsep}}}{$100$} &  \\
Sex & Male  & $266$ & \multicolumn{1}{@{\hspace{\tabcolsep}(}r@{)\hspace{\tabcolsep}}}{$62$} & $131$ & \multicolumn{1}{@{\hspace{\tabcolsep}(}r@{)\hspace{\tabcolsep}}}{$65$} & $397$ \\
 & Female  & $161$ & \multicolumn{1}{@{\hspace{\tabcolsep}(}r@{)\hspace{\tabcolsep}}}{$38$} & $\phantom{0}72$ & \multicolumn{1}{@{\hspace{\tabcolsep}(}r@{)\hspace{\tabcolsep}}}{$35$} & $233$ \\
 & \\  %  &  & \multicolumn{1}{@{\hspace{\tabcolsep}(}r@{)\hspace{\tabcolsep}}}{$100$} &  & \multicolumn{1}{@{\hspace{\tabcolsep}(}r@{)\hspace{\tabcolsep}}}{$100$} &  \\
Age at recruitment (years) & [26.7,55)  & $118$ & \multicolumn{1}{@{\hspace{\tabcolsep}(}r@{)\hspace{\tabcolsep}}}{$28$} & $\phantom{0}37$ & \multicolumn{1}{@{\hspace{\tabcolsep}(}r@{)\hspace{\tabcolsep}}}{$18$} & $155$ \\
 & [55,65)  & $161$ & \multicolumn{1}{@{\hspace{\tabcolsep}(}r@{)\hspace{\tabcolsep}}}{$38$} & $\phantom{0}88$ & \multicolumn{1}{@{\hspace{\tabcolsep}(}r@{)\hspace{\tabcolsep}}}{$43$} & $249$ \\
 & [65,86.8]  & $148$ & \multicolumn{1}{@{\hspace{\tabcolsep}(}r@{)\hspace{\tabcolsep}}}{$35$} & $\phantom{0}78$ & \multicolumn{1}{@{\hspace{\tabcolsep}(}r@{)\hspace{\tabcolsep}}}{$38$} & $226$ \\
 & \\ %  &  & \multicolumn{1}{@{\hspace{\tabcolsep}(}r@{)\hspace{\tabcolsep}}}{$100$} &  & \multicolumn{1}{@{\hspace{\tabcolsep}(}r@{)\hspace{\tabcolsep}}}{$100$} &  \\
Country & Czech Republic  & $230$ & \multicolumn{1}{@{\hspace{\tabcolsep}(}r@{)\hspace{\tabcolsep}}}{$54$} & $\phantom{0}95$ & \multicolumn{1}{@{\hspace{\tabcolsep}(}r@{)\hspace{\tabcolsep}}}{$47$} & $325$ \\
 & Russia  & $167$ & \multicolumn{1}{@{\hspace{\tabcolsep}(}r@{)\hspace{\tabcolsep}}}{$39$} & $105$ & \multicolumn{1}{@{\hspace{\tabcolsep}(}r@{)\hspace{\tabcolsep}}}{$52$} & $272$ \\
 & Romania  & $\phantom{0}30$ & \multicolumn{1}{@{\hspace{\tabcolsep}(}r@{)\hspace{\tabcolsep}}}{$\phantom{0}7$} & $\phantom{00}3$ & \multicolumn{1}{@{\hspace{\tabcolsep}(}r@{)\hspace{\tabcolsep}}}{$\phantom{0}1$} & $\phantom{0}33$ \\
 & \\ %  &  & \multicolumn{1}{@{\hspace{\tabcolsep}(}r@{)\hspace{\tabcolsep}}}{$100$} &  & \multicolumn{1}{@{\hspace{\tabcolsep}(}r@{)\hspace{\tabcolsep}}}{$100$} &  \\
BMI (kg/m$^2$) & [17.2,25)  & $\phantom{0}97$ & \multicolumn{1}{@{\hspace{\tabcolsep}(}r@{)\hspace{\tabcolsep}}}{$23$} & $\phantom{0}71$ & \multicolumn{1}{@{\hspace{\tabcolsep}(}r@{)\hspace{\tabcolsep}}}{$35$} & $168$ \\
 & [25,30)  & $184$ & \multicolumn{1}{@{\hspace{\tabcolsep}(}r@{)\hspace{\tabcolsep}}}{$43$} & $\phantom{0}82$ & \multicolumn{1}{@{\hspace{\tabcolsep}(}r@{)\hspace{\tabcolsep}}}{$40$} & $266$ \\
 & [30,58.5]  & $143$ & \multicolumn{1}{@{\hspace{\tabcolsep}(}r@{)\hspace{\tabcolsep}}}{$33$} & $\phantom{0}50$ & \multicolumn{1}{@{\hspace{\tabcolsep}(}r@{)\hspace{\tabcolsep}}}{$25$} & $193$ \\
 & missing  & $\phantom{00}3$ & \multicolumn{1}{@{\hspace{\tabcolsep}(}r@{)\hspace{\tabcolsep}}}{$\phantom{0}1$} & $\phantom{00}0$ & \multicolumn{1}{@{\hspace{\tabcolsep}(}r@{)\hspace{\tabcolsep}}}{$\phantom{0}0$} & $\phantom{00}3$ \\
 & \\ %  &  & \multicolumn{1}{@{\hspace{\tabcolsep}(}r@{)\hspace{\tabcolsep}}}{$100$} &  & \multicolumn{1}{@{\hspace{\tabcolsep}(}r@{)\hspace{\tabcolsep}}}{$100$} &  \\
Smoking & Never smoker  & $213$ & \multicolumn{1}{@{\hspace{\tabcolsep}(}r@{)\hspace{\tabcolsep}}}{$50$} & $\phantom{0}92$ & \multicolumn{1}{@{\hspace{\tabcolsep}(}r@{)\hspace{\tabcolsep}}}{$45$} & $305$ \\
 & Former smoker  & $108$ & \multicolumn{1}{@{\hspace{\tabcolsep}(}r@{)\hspace{\tabcolsep}}}{$25$} & $\phantom{0}54$ & \multicolumn{1}{@{\hspace{\tabcolsep}(}r@{)\hspace{\tabcolsep}}}{$27$} & $162$ \\
 & Current smoker  & $106$ & \multicolumn{1}{@{\hspace{\tabcolsep}(}r@{)\hspace{\tabcolsep}}}{$25$} & $\phantom{0}57$ & \multicolumn{1}{@{\hspace{\tabcolsep}(}r@{)\hspace{\tabcolsep}}}{$28$} & $163$ \\
 & \\ %  &  & \multicolumn{1}{@{\hspace{\tabcolsep}(}r@{)\hspace{\tabcolsep}}}{$100$} &  & \multicolumn{1}{@{\hspace{\tabcolsep}(}r@{)\hspace{\tabcolsep}}}{$100$} &  \\
Diabetic & Yes  & $\phantom{0}65$ & \multicolumn{1}{@{\hspace{\tabcolsep}(}r@{)\hspace{\tabcolsep}}}{$15$} & $\phantom{0}39$ & \multicolumn{1}{@{\hspace{\tabcolsep}(}r@{)\hspace{\tabcolsep}}}{$19$} & $104$ \\
 & No  & $362$ & \multicolumn{1}{@{\hspace{\tabcolsep}(}r@{)\hspace{\tabcolsep}}}{$85$} & $164$ & \multicolumn{1}{@{\hspace{\tabcolsep}(}r@{)\hspace{\tabcolsep}}}{$81$} & $526$ \\
 & \\ %  &  & \multicolumn{1}{@{\hspace{\tabcolsep}(}r@{)\hspace{\tabcolsep}}}{$100$} &  & \multicolumn{1}{@{\hspace{\tabcolsep}(}r@{)\hspace{\tabcolsep}}}{$100$} &  \\
Hypertension & Yes  & $233$ & \multicolumn{1}{@{\hspace{\tabcolsep}(}r@{)\hspace{\tabcolsep}}}{$55$} & $103$ & \multicolumn{1}{@{\hspace{\tabcolsep}(}r@{)\hspace{\tabcolsep}}}{$51$} & $336$ \\
 & No  & $193$ & \multicolumn{1}{@{\hspace{\tabcolsep}(}r@{)\hspace{\tabcolsep}}}{$45$} & $100$ & \multicolumn{1}{@{\hspace{\tabcolsep}(}r@{)\hspace{\tabcolsep}}}{$49$} & $293$ \\
 & missing  & $\phantom{00}1$ & \multicolumn{1}{@{\hspace{\tabcolsep}(}r@{)\hspace{\tabcolsep}}}{$\phantom{0}0$} & $\phantom{00}0$ & \multicolumn{1}{@{\hspace{\tabcolsep}(}r@{)\hspace{\tabcolsep}}}{$\phantom{0}0$} & $\phantom{00}1$ \\
 & \\ %  &  & \multicolumn{1}{@{\hspace{\tabcolsep}(}r@{)\hspace{\tabcolsep}}}{$100$} &  & \multicolumn{1}{@{\hspace{\tabcolsep}(}r@{)\hspace{\tabcolsep}}}{$100$} &  \\
Stage & I  & $269$ & \multicolumn{1}{@{\hspace{\tabcolsep}(}r@{)\hspace{\tabcolsep}}}{$63$} & $\phantom{0}38$ & \multicolumn{1}{@{\hspace{\tabcolsep}(}r@{)\hspace{\tabcolsep}}}{$19$} & $307$ \\
 & II  & $\phantom{0}38$ & \multicolumn{1}{@{\hspace{\tabcolsep}(}r@{)\hspace{\tabcolsep}}}{$\phantom{0}9$} & $\phantom{0}10$ & \multicolumn{1}{@{\hspace{\tabcolsep}(}r@{)\hspace{\tabcolsep}}}{$\phantom{0}5$} & $\phantom{0}48$ \\
 & III  & $\phantom{0}56$ & \multicolumn{1}{@{\hspace{\tabcolsep}(}r@{)\hspace{\tabcolsep}}}{$13$} & $\phantom{0}54$ & \multicolumn{1}{@{\hspace{\tabcolsep}(}r@{)\hspace{\tabcolsep}}}{$27$} & $110$ \\
 & IV  & $\phantom{0}63$ & \multicolumn{1}{@{\hspace{\tabcolsep}(}r@{)\hspace{\tabcolsep}}}{$15$} & $101$ & \multicolumn{1}{@{\hspace{\tabcolsep}(}r@{)\hspace{\tabcolsep}}}{$50$} & $164$ \\
 & missing  & $\phantom{00}1$ & \multicolumn{1}{@{\hspace{\tabcolsep}(}r@{)\hspace{\tabcolsep}}}{$\phantom{0}0$} & $\phantom{00}0$ & \multicolumn{1}{@{\hspace{\tabcolsep}(}r@{)\hspace{\tabcolsep}}}{$\phantom{0}0$} & $\phantom{00}1$ \\
 & \\ %  &  & \multicolumn{1}{@{\hspace{\tabcolsep}(}r@{)\hspace{\tabcolsep}}}{$100$} &  & \multicolumn{1}{@{\hspace{\tabcolsep}(}r@{)\hspace{\tabcolsep}}}{$100$} &  \\
Grade & I  & $\phantom{0}73$ & \multicolumn{1}{@{\hspace{\tabcolsep}(}r@{)\hspace{\tabcolsep}}}{$17$} & $\phantom{00}7$ & \multicolumn{1}{@{\hspace{\tabcolsep}(}r@{)\hspace{\tabcolsep}}}{$\phantom{0}3$} & $\phantom{0}80$ \\
 & II  & $173$ & \multicolumn{1}{@{\hspace{\tabcolsep}(}r@{)\hspace{\tabcolsep}}}{$41$} & $\phantom{0}53$ & \multicolumn{1}{@{\hspace{\tabcolsep}(}r@{)\hspace{\tabcolsep}}}{$26$} & $226$ \\
 & III  & $\phantom{0}67$ & \multicolumn{1}{@{\hspace{\tabcolsep}(}r@{)\hspace{\tabcolsep}}}{$16$} & $\phantom{0}50$ & \multicolumn{1}{@{\hspace{\tabcolsep}(}r@{)\hspace{\tabcolsep}}}{$25$} & $117$ \\
 & IV  & $\phantom{0}15$ & \multicolumn{1}{@{\hspace{\tabcolsep}(}r@{)\hspace{\tabcolsep}}}{$\phantom{0}4$} & $\phantom{0}18$ & \multicolumn{1}{@{\hspace{\tabcolsep}(}r@{)\hspace{\tabcolsep}}}{$\phantom{0}9$} & $\phantom{0}33$ \\
 & missing  & $\phantom{0}99$ & \multicolumn{1}{@{\hspace{\tabcolsep}(}r@{)\hspace{\tabcolsep}}}{$23$} & $\phantom{0}75$ & \multicolumn{1}{@{\hspace{\tabcolsep}(}r@{)\hspace{\tabcolsep}}}{$37$} & $174$ \\
 & \\ %  &  & \multicolumn{1}{@{\hspace{\tabcolsep}(}r@{)\hspace{\tabcolsep}}}{$100$} &  & \multicolumn{1}{@{\hspace{\tabcolsep}(}r@{)\hspace{\tabcolsep}}}{$100$} &  \\
Histology & Conventional RCC  & $357$ & \multicolumn{1}{@{\hspace{\tabcolsep}(}r@{)\hspace{\tabcolsep}}}{$84$} & $161$ & \multicolumn{1}{@{\hspace{\tabcolsep}(}r@{)\hspace{\tabcolsep}}}{$79$} & $518$ \\
 & Papillary RCC  & $\phantom{0}41$ & \multicolumn{1}{@{\hspace{\tabcolsep}(}r@{)\hspace{\tabcolsep}}}{$10$} & $\phantom{0}10$ & \multicolumn{1}{@{\hspace{\tabcolsep}(}r@{)\hspace{\tabcolsep}}}{$\phantom{0}5$} & $\phantom{0}51$ \\
 & Chromophobe RCC  & $\phantom{0}11$ & \multicolumn{1}{@{\hspace{\tabcolsep}(}r@{)\hspace{\tabcolsep}}}{$\phantom{0}3$} & $\phantom{00}5$ & \multicolumn{1}{@{\hspace{\tabcolsep}(}r@{)\hspace{\tabcolsep}}}{$\phantom{0}2$} & $\phantom{0}16$ \\
 & Other  & $\phantom{0}14$ & \multicolumn{1}{@{\hspace{\tabcolsep}(}r@{)\hspace{\tabcolsep}}}{$\phantom{0}3$} & $\phantom{00}4$ & \multicolumn{1}{@{\hspace{\tabcolsep}(}r@{)\hspace{\tabcolsep}}}{$\phantom{0}2$} & $\phantom{0}18$ \\
 & Unknown  & $\phantom{00}4$ & \multicolumn{1}{@{\hspace{\tabcolsep}(}r@{)\hspace{\tabcolsep}}}{$\phantom{0}1$} & $\phantom{0}23$ & \multicolumn{1}{@{\hspace{\tabcolsep}(}r@{)\hspace{\tabcolsep}}}{$11$} & $\phantom{0}27$ \\
 & \\ %  &  & \multicolumn{1}{@{\hspace{\tabcolsep}(}r@{)\hspace{\tabcolsep}}}{$100$} &  & \multicolumn{1}{@{\hspace{\tabcolsep}(}r@{)\hspace{\tabcolsep}}}{$100$} &  \\
 & missing  & $427$ & \multicolumn{1}{@{\hspace{\tabcolsep}(}r@{)\hspace{\tabcolsep}}}{$100$} & $203$ & \multicolumn{1}{@{\hspace{\tabcolsep}(}r@{)\hspace{\tabcolsep}}}{$100$} & $630$ \\
\bottomrule 
\end{tabular}

\end{table}

\clearpage 
%% HR table
\begin{table}
\caption{Hazard ratios (HR) [95\% confidence intervals (CI)] for risk of all 
cause and cause specific mortality by season-adjusted categories of 25(OH)D$_3$ 
concentration.}
\begin{tabular}{lrrrrrrrr}
\toprule
& & &\multicolumn{3}{c}{minimally adjusted$^\dag$} & 
\multicolumn{3}{c}{adjusted$^\ddag$} \\
 \cmidrule(r){4-6} \cmidrule(l){7-9} 
& D$_3$ category & $N_{\text{deaths}}$ & HR & [95\% CI] & $p$ & HR & [95\% CI] 
& 
$p$ \\
\midrule
all cause&1&63&1.00&&.015$^*$&1.00&&.03$^*$\\
&2&56&1.14&[0.69, 1.90]&&1.12&[0.67, 1.87]&\\
&3&46&0.81&[0.48, 1.37]&&0.86&[0.51, 1.44]&\\
&4&38&0.57&[0.34, 0.97]&&0.59&[0.35, 1.00]&\\
\\ RCC&1&42&1.00&&.56$^\S$&1.00&&.53$^\S$\\
&2&43&1.32&[0.76, 2.31]&&1.30&[0.74, 2.27]&\\
&3&36&0.96&[0.54, 1.70]&&1.01&[0.57, 1.79]&\\
&4&31&0.68&[0.38, 1.20]&&0.70&[0.39, 1.24]&\\
\\ non-RCC&1&21&1.00&&&1.00&&\\
&2&13&0.79&[0.36, 1.75]&&0.76&[0.34, 1.70]&\\
&3&10&0.52&[0.21, 1.28]&&0.55&[0.23, 1.35]&\\
&4&7&0.36&[0.14, 0.92]&&0.36&[0.14, 0.91]&\\

\bottomrule
\end{tabular}
\newline
{\footnotesize 
$^\dag$Stratified by country, and adjusted for stage, age at recruitment, and 
sex \newline
$^\ddag$Additionally adjusted for BMI (kg/m$^2$), smoking status, cigarettes 
per day, alcohol drinking status, and alcohol intake per day (mL) \newline
$^*p$-values for the all-cause models are from tests against the null 
hypothesis that the 25(OH)D$_3$ coefficients are identically 0. 
\newline
$^{\S}p$-values for the competing risks model are from tests against the null 
hypothesis of no heterogeneity of the coefficients by cause of death (RCC 
versus non-RCC).
}
\end{table}
\clearpage

%% continuously varying HR
\begin{figure}
 \centering
\caption*{Figure 1: Hazard ratio for all cause mortality among RCC cases as a 
function of 
circulating concentration of 25(OH)D$_3$ at diagnosis, relative to a 
concentration of 50~nmol/L. 25(OH)D$_3$ was modeled using restricted cubic 
splines with knots at the 10th, 33rd, 67th, and 90th percentiles of its 
distribution. Estimates were derived from a Cox model stratified by country of 
recruitment, and adjusted for stage, age at recruitment, sex, and seasonality 
(sine and cosine functions of day of blood draw). Solid and 
dashed lines represent the maximum pseudolikelihood estimates and 95\% 
confidence intervals respectively. The translucent lines are 1000 draws from 
the multivariate normal distribution defined by the maximum pseudolikelihood 
estimates and their variance covariance matrix, and thus give an indication of 
the posterior density for the hazard ratio under a uniform prior on the 
regression coefficients. The ``rug plot'' shows the observed distribution of 
25(OH)D$_3$.}
\includegraphics[width=12cm]{../analysis/output/g04_hr.pdf}
\end{figure}

%% Interactions with various factors
\begin{figure}
 \centering
\caption*{Supplementary Figure 1: Hazard ratios (HR) and 95\% confidence intervals 
(CI) for a 
doubling in season-adjusted 25(OH)D$_3$ concentration by potential effect 
modifiers. \footnotesize{Estimates derived from Cox models stratified by country 
of recruitment, and adjusted for stage, age at recruitment, and sex. $p$-values 
are from Wald tests of the interaction terms.}}
\includegraphics[width=14cm]{../analysis/output/g09_interactions.pdf}
\end{figure}

%% Short report, so don't worry about the rest of the tables and figures
\end{document}



%% Model based survival by d3 and stage
\begin{figure}
 \centering 
\caption{\textbf{Model based survival function by categories of 
circulating 
vitamin D$_3$ and stage.} \footnotesize{Estimates derived from flexible 
parametric survival models assuming proportional hazards for D$_3$ categories, 
no 
interaction between stage and D$_3$ categories, and a diagnosis age of 60y.}} 
\includegraphics[width=14cm]{../analysis/output/g06_surv_d3_stage.pdf}
\end{figure}

\clearpage
%% HR table for relapse
\begin{table}
\caption*{Supplementary Table 1: Hazard ratios (HR) [95\% confidence intervals 
(CI)] for risk of 
relapse free survival, and from a competing risks model of relapse versus 
death, by categories of circulating vitamin D$_3$ concentration.}
\begin{tabular}{lrrrrrrrr}
\toprule
& & &\multicolumn{3}{c}{minimally adjusted$^\dag$} & 
\multicolumn{3}{c}{adjusted$^\ddag$} \\
 \cmidrule(r){4-6} \cmidrule(l){7-9} 
& D$_3$ category & $N_{\text{events}}^{\P}$ & HR & [95\% CI] & $p$ & HR & [95\% 
CI] 
& 
$p$ \\
\midrule
Relapse or Death&1&23&1.00&&.012$^*$&1.00&&.029$^*$\\
&2&25&1.09&[0.60, 1.97]&&1.08&[0.59, 1.97]&\\
&3&14&0.61&[0.31, 1.21]&&0.66&[0.33, 1.32]&\\
&4&13&0.51&[0.26, 1.03]&&0.50&[0.25, 1.02]&\\
\\ Relapse&1&8&1.00&&.029$^\S$&1.00&&.025$^\S$\\
&2&16&2.13&[0.89, 5.07]&&2.12&[0.88, 5.11]&\\
&3&8&0.98&[0.36, 2.64]&&1.07&[0.39, 2.92]&\\
&4&11&1.32&[0.52, 3.37]&&1.33&[0.52, 3.42]&\\
\\ Death&1&15&1.00&&&1.00&&\\
&2&9&0.56&[0.23, 1.33]&&0.55&[0.23, 1.32]&\\
&3&6&0.42&[0.16, 1.11]&&0.44&[0.16, 1.17]&\\
&4&2&0.12&[0.03, 0.51]&&0.11&[0.02, 0.48]&\\

\bottomrule
\end{tabular}
\newline
{\footnotesize 
$^{\P}$Number of deaths and/or relapses occurring within the randomly selected 
subcohort. Deaths occurring outside the subcohort are not included in this 
analysis, so there are fewer events than in the main analysis presented in 
Table 2. \newline
$^\dag$Stratified by country, and adjusted for stage, age at recruitment, and 
sex \newline
$^\ddag$Additionally adjusted for BMI (kg/m$^2$), smoking status, cigarettes 
per day, alcohol drinking status, and alcohol intake per day (mL) \newline
$^*p$-values for the relapse free models are from tests against the null 
hypothesis that the vitamin D$_3$ coefficients are identically 0. \newline
$^{\S}p$-values for the competing risks model are from tests against the null 
hypothesis of no difference between the coefficients for relapse versus death.
}
\end{table}

%% HR table adjusting for grade
\begin{table}
\centering
\caption*{Supplementary Table 2: Hazard ratios$^\dag$ (HR) [95\% confidence 
intervals 
(CI)] for risk of all 
cause and cause specific mortality by categories of circulating vitamin D$_3$ 
concentration, additionally adjusting for grade (participants with missing 
grade excluded).}
\begin{tabular}{lrrrrrrrr}
\toprule
& D$_3$ category & $N_{\text{events}}^{\P}$ & HR & [95\% CI] & $p$ \\
\midrule
all cause&1&43&1.00&&.04$^*$\\
&2&33&1.10&[0.57, 2.13]&\\
&3&28&0.89&[0.47, 1.69]&\\
&4&22&0.52&[0.26, 1.02]&\\
\\ RCC&1&27&1.00&&.69$^\S$\\
&2&24&1.35&[0.65, 2.82]&\\
&3&20&1.02&[0.49, 2.11]&\\
&4&17&0.63&[0.29, 1.34]&\\
\\ non-RCC&1&16&1.00&&\\
&2&9&0.71&[0.27, 1.88]&\\
&3&8&0.68&[0.26, 1.79]&\\
&4&5&0.34&[0.11, 1.05]&\\

\bottomrule
\end{tabular}
\newline
{\footnotesize 
$^\dag$Stratified by country, and adjusted for stage, age at recruitment, sex, 
BMI (kg/m$^2$), smoking status, cigarettes per day, alcohol drinking status,
alcohol intake per day (mL), and grade (with participants with missing grade 
excluded).\newline
$^*p$-values for the relapse free models are from tests against the null 
hypothesis that the vitamin D$_3$ coefficients are identically 0. \newline
$^{\S}p$-values for the competing risks model are from tests against the null 
hypothesis of no difference between the coefficients for relapse versus death.
}
\end{table}
\clearpage


\end{document}

