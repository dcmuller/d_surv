%% Vitamin D and RCC survival
%% draft manuscript

\documentclass[a4paper,11pt]{article}
%\usepackage[numbers,comma,square,sort&compress]{natbib}
\usepackage{cite}
\renewcommand{\citeleft}{(}
\renewcommand{\citeright}{)}
\usepackage{setspace}

\usepackage[left=2cm,right=2cm,top=2cm,bottom=2cm]{geometry}
\usepackage{graphicx}
\usepackage{amsmath}
\usepackage{amssymb}


\usepackage{threeparttable}
\usepackage{color, soul}
\usepackage{booktabs}
\usepackage{multicol}
\usepackage{pdflscape}
\usepackage{hyperref}
\hypersetup{
  colorlinks=true,
  urlcolor=black,
  citecolor=black,
  linkcolor=black,
}

%% use palatino font
\usepackage{mathpazo}
\linespread{1.15}

%% custom captions
\usepackage[labelformat=empty]{caption}

%% indenting and paragraphing
\parskip 10pt
\parindent 25pt

%\renewcommand{\cite}{\citep}

%%%%%%%%%%%%%%%%%%%%%%%%%%%%%%%%%%%%%%%%%%%%%%%%%%%%%%%%%%%%%%%%%%%%
%% #1 end preamble. Begin document proper
%%%%%%%%%%%%%%%%%%%%%%%%%%%%%%%%%%%%%%%%%%%%%%%%%%%%%%%%%%%%%%%%%%%%

\begin{document}
\doublespace
\noindent \textbf{Title:} Circulating 25-Hydroxyvitamin D$_3$ and 
survival after diagnosis with kidney cancer. 

\noindent \textbf{Authors:} David C Muller (1), Ghislaine Scelo (1),  David 
Zaridze (2), Vladimir Janout (3), Ivana Holcatova (4), Lenka 
Foretova/Marie Navratilova (5), Dana Mates (6), {\O}ivind Midttun (7), 
Per Magne Ueland (8,9), Paul Brennan (1), Mattias Johansson (1).

{\footnotesize 
\noindent
1. International Agency for Research on Cancer (IARC), Lyon, France \newline
2. Russian N.N. Blokhin Cancer Research Centre, Moscow, Russian Federation 
\newline
3. Department of Preventive Medicine, Faculty of Medicine, Palacky University, 
Olomouc, Czech Republic \newline
4. Charles University in Prague, First Faculty of Medicine, Institute of 
Hygiene 
and Epidemiology, Prague, Czech Republic \newline
5. Department of Cancer Epidemiology and Genetics, Masaryk Memorial Cancer 
Institute, Brno, Czech Republic \newline
6. National Institute of Public Health, Bucharest, Romania
7. Bevital AS, Bergen, Norway \newline
8. Department of Clinical Science, University of Bergen, Norway \newline
9. Laboratory of Clinical Biochemistry, Haukeland University Hospital, Bergen, 
Norway
}

\textbf{Corresponding authors:} 




%%%%%%%%%%%%%%%%%%%%%%%%%%%%%%%%%%%%%%%%%%%%%%%%%%%%%%%%%%%%%%%%%%%%
%% #2 abstract
%%%%%%%%%%%%%%%%%%%%%%%%%%%%%%%%%%%%%%%%%%%%%%%%%%%%%%%%%%%%%%%%%%%%
\begin{abstract}
\noindent \textbf{Background:} We evaluated whether concentrations of vitamin 
D at diagnosis of renal cell carcinoma (RCC) is associated with prognosis.  
\newline 
\textbf{Methods:} We conducted a case-cohort study of 630 RCC cases from a 
multi-centre case-control study in eastern Europe. Vitamin D was assessed as 
25-Hydroxyvitamin D$_3$ [25(OH)D$_3$], and we used weighted Cox models to estimate 
hazard ratios (HR) and 95\% confidence intervals (CI) by categories of 
season-adjusted 25(OH)D$_3$.\newline 
\textbf{Results:} Higher concentrations of 25(OH)D$_3$ 
were associated with lower risk of death after adjusting for stage, age, sex, 
and country (HR highest versus lowest category 0.57, 95\% CI [0.34, 0.97]). The 
inverse associations of 25(OH)D$_3$ with death were most notable among those who 
died from non-RCC causes and those diagnosed with early stage disease. 
\newline 
\textbf{Conclusions:} 25(OH)D$_3$ concentration at diagnosis 
of RCC was inversely associated with all-cause mortality rates, but not specifically 
with RCC outcome. \newline

\end{abstract}

%%%%%%%%%%%%%%%%%%%%%%%%%%%%%%%%%%%%%%%%%%%%%%%%%%%%%%%%%%%%%%%%%%%%
%% #3 introduction
%%%%%%%%%%%%%%%%%%%%%%%%%%%%%%%%%%%%%%%%%%%%%%%%%%%%%%%%%%%%%%%%%%%%
\clearpage


\section*{Introduction}
Each year more than 300,000 new cases of kidney cancer are diagnosed worldwide, 
leading to approximately 130,000 deaths \cite{ferlay_cancer_2013}. The prognosis 
is strongly dependent on stage at diagnosis, with around 90\% of stage I 
patients alive five years after diagnosis, compared with only 10\% of stage IV 
patients \cite{CRUK_kidney_2014}.

Little is known about factors influencing survival after diagnosis of kidney 
cancer, apart from tumor stage and grade. We recently investigated circulating 
vitamin D and kidney cancer onset and survival in a prospective epidemiological 
cohort where blood samples were collected at entry to the cohort, an average 
of 7 years before diagnosis \cite{muller_circulating_2014}. We reported an 
inverse association between vitamin D concentrations and risk of subsequent 
kidney cancer, an observation consistent with another recent study in which 
investigators had estimated vitamin D levels \cite{joh_predicted_2013}. 
Further, we observed an increased rate of death after kidney cancer diagnosis 
for both low and high concentrations of vitamin D. Available data did not allow 
thorough analysis of cause-specific mortality, nor were we able to adjust for 
stage. This observation prompted us to investigate whether vitamin D 
concentrations in blood at kidney cancer diagnosis are associated with 
subsequent survival, and if such an association is independent of stage or 
other prognostic factors.   

%%%%%%%%%%%%%%%%%%%%%%%%%%%%%%%%%%%%%%%%%%%%%%%%%%%%%%%%%%%%%%%%%%%%
%% #3 methods 
%%%%%%%%%%%%%%%%%%%%%%%%%%%%%%%%%%%%%%%%%%%%%%%%%%%%%%%%%%%%%%%%%%%%
\section*{Methods}
\subsection*{The K2 study}
Participants included patients who were above 18 years of age and diagnosed 
with kidney cancer in one of 4 participating centres in Czech Republic, 1 
center in Romania, and 1 center in Russia. We gave 
participants a standardized face-to-face short lifestyle 
questionnaire covering socio-demographic characteristics, anthropometric 
measures, medical history, family history, and tobacco and alcohol use. 
Clinical and pathological data were abstracted from medical charts and 
pathological reports. A majority of participants underwent nephrectomy and the 
tumor was histologically confirmed. Follow-up for outcome (relapse, vital 
status, and cause of death where relevant) was performed every 6 to 12 months 
after diagnosis, using passive follow-up methods where possible (with 
confirmation of vital status through active follow-up methods in case of 
uncertainties), and active follow-up methods when no linkage to databases was 
possible. The study protocol was approved by the institutional review boards of the 
International Agency for Research on Cancer and all collaborating institutions, and 
we obtained written informed consent from all participants. 

\subsection*{Case-cohort sampling}
Among 2330 participants with questionnaire data available and a diagnosis of 
renal cell carcinoma (RCC), we excluded 125 participants 
with no plasma sample available, 1005 participants with no follow-up data at 
the time of this project, 5 participants with inconsistencies in reference 
dates, and 7 participants with no information on stage. From the 1188 
remaining, we randomly selected 500 participants at baseline (the subcohort). 
We also included all participants who died during follow-up that were not 
randomly selected in to the subcohort ($N=93$), as well as 37 stage IV patients 
that had survived and were not randomly selected. Hence a total of 
630 participants diagnosed with kidney cancer were included in the study.

\subsection*{Biosample processing and biochemical analysis}
Venous blood was obtained before or at the time of the nephrectomy, prior to 
any treatment. Blood was collected in vacutainers containing 
ethylenediaminetetraacetic acid (EDTA), and processed as rapidly as possible 
(usually within two hours). Plasma samples were stored at $-80^{\circ}$C, 
except in Ceske Budejovice, Czech Republic, where samples were stored at 
$-20^{\circ}$C. All samples were transported at $-80^{\circ}$C to IARC for 
long-term storage at $-150^{\circ}$C. Samples underwent a single thawing cycle 
for aliquoting of 400uL for shipment to the Bevital laboratory in Bergen, 
Norway, for analysis. Liquid chromatography coupled to tandem 
mass spectrometry was used to separately analyse vitamin D as 25(OH)D$_2$ and 
25(OH)D$_3$ \cite{midttun_determination_2011}. 25(OH)D$_2$ was undetectable in 
the majority of samples, so our analyses focus on 25(OH)D$_3$. The lab is DEQAS 
(www.deqas.org) certified.

\subsection*{Statistical analysis}
To adjust for seasonal variation, we modeled the expected log$_2$ 25(OH)D$_3$ 
concentration as a periodic function of day of blood draw using a pair of sine 
and cosine functions. To create season-adjusted categories, we grouped the 
residuals from this model at quartiles of their distribution among the randomly 
selected subcohort. We used Cox proportional hazards models with time since 
diagnosis (recruitment) as the time scale to estimate hazard ratios (HR) and 
95\% confidence intervals (CI) for all-cause mortality by the season-adjusted 
categories of 25(OH)D$_3$. We also modeled 25(OH)D$_3$ continuously using 
restricted cubic splines with knots at its 10th, 33rd, 67th, and 90th 
percentiles, explicitly adjusting for seasonality by including the pair of sine 
and cosine functions. HRs for cause-specific mortality were calculated in a 
competing risks model using the data augmentation method 
\cite{lunn_applying_1995}. To account for the case-cohort design 
\cite{prentice_case-cohort_1986}, we used Barlow's method to weight the 
likelihood and computed robust variance estimates \cite{barlow_robust_1994, 
barlow_analysis_1999}. We investigated potential effect modification by fitting 
interactions between season-adjusted log$_2$ 25(OH)D$_3$ and various factors. 
All models included stage, age at recruitment, and sex as covariates, with the 
baseline hazard stratified by country of recruitment. We additionally adjusted 
for body mass index (BMI, kg/m$^2$), smoking status (never, former, current), 
and alcohol drinking status (never, former, current).
All $p$-values are two sided, and were calculated using the Wald test. 
Statistical analyses were performed using Stata 12.1 for Linux (Stata 
Corporation, College Station, Texas, US) and R version 3.1.1 \cite{r_2014}.


%%%%%%%%%%%%%%%%%%%%%%%%%%%%%%%%%%%%%%%%%%%%%%%%%%%%%%%%%%%%%%%%%%%%
%% #4 results
%%%%%%%%%%%%%%%%%%%%%%%%%%%%%%%%%%%%%%%%%%%%%%%%%%%%%%%%%%%%%%%%%%%%
\section*{Results}
Demographic and clinical characteristics of the study sample by vital status at 
the end of follow-up are presented in Table~1. The sample included a higher 
proportion of men (63\%) than women, and was predominantly recruited from the 
Czech Republic (52\%) and Russia (43\%), with only 5\% or participants 
recruited in Romania. Those participants who survived to the end of follow-up 
had a similar age distribution to those who died during follow-up. 518 
of the 630 cases (82\%) were conventional RCC. 50\% of deaths occurred among 
participants with a stage IV tumor, and 15\% of those surviving to the end of 
follow-up had stage IV diagnoses. In contrast, 72 \% of those surviving to the 
end of follow-up were diagnosed with stage I-II disease.

Hazard of death from any cause was inversely associated with 
circulating concentrations of 25(OH)D$_3$ (Table~2). After adjusting for 
stage, age, and sex, the hazard was 43\% lower among those in the highest 
compared to the lowest group of seasonally-adjusted concentration 
(HR$_{4\text{vs}1}$ 0.57, 95\% CI [0.34, 0.97]). Although no statistical 
evidence for heterogeneity by cause of death was noted 
($P_\text{heterogeneity}$ 0.53), we estimated HR$_{4\text{vs}1}$ of 0.77 
(95\% CI 0.39, 1.24) for RCC specific death, and 0.36 (95\% CI 0.14, 0.91) for 
non-RCC causes of death, suggesting that this association was not specific to 
RCC death (Table~2). The HR for continuously varying 25(OH)D$_3$ (relative to a 
concentration of 50 nmol/L) is presented in Figure~1. These estimates 
corroborate those in Table~2, suggesting a monotonic inverse association between 
25(OH)D$_3$ and hazard of death.

Figure~2 presents HRs for a doubling in seasonally adjusted 25(OH)D$_3$ 
concentration separately by categories of several potential effect modifiers. 
The estimated magnitude of the association was consistent by, sex, stage, 
histology, history of diabetes, smoking status, and alcohol intake status. 
There was some indication that the association might be stronger among those 
diagnosed at age 65 years or older, those with a history of hypertension, those 
with higher BMI, and those diagnosed with stage I or II RCC, but there was 
little statistical evidence of interaction with any of these factors. 

%%%%%%%%%%%%%%%%%%%%%%%%%%%%%%%%%%%%%%%%%%%%%%%%%%%%%%%%%%%%%%%%%%%%
%% #4 discussion
%%%%%%%%%%%%%%%%%%%%%%%%%%%%%%%%%%%%%%%%%%%%%%%%%%%%%%%%%%%%%%%%%%%%
\section*{Discussion}
We investigated whether differences in circulating concentrations of 
25(OH)D$_3$ at the time of diagnosis of RCC were associated with all-cause and 
RCC-specific survival. We observed that higher concentrations of 25(OH)D$_3$ 
were associated with a lower rate of death, but that this association was not 
restricted to RCC specific death. We also observed an indication this 
association might be somewhat stronger for those with a 
history of hypertension, advanced age at diagosis, or early stage disease.

We recently studied circulating 25(OH)D$_3$ and RCC risk of RCC in a prospective
case-control study nested within the EPIC cohort \cite{muller_circulating_2014}.
This analysis indicated an inverse association between 
25(OH)D$_3$ and risk of RCC as well as a non-linear U-shaped association 
between pre-diagnostic 25(OH)D$_3$ and all-cause mortality after diagnosis 
of RCC. This observation prompted us to conduct the current analysis in newly 
diagnosed RCC cases with complete information on disease stage 
and cause of death. Results from the present study are not completely 
consistent with these initial findings from EPIC. In particular, we found no 
evidence of increased rate of death among patients with high 25(OH)D$_3$ at 
diagnosis, but rather an inverse association between 25(OH)D$_3$ and all-cause 
mortality across the range of observed concentrations.

Many studies have investigated circulating vitamin D and all-cause 
mortality in general populations. Consistent with our observation, many of these
have reported high risk of death for people with low concentrations 
\cite{schottker_strong_2013, virtanen_association_2011, hutchinson_low_2010, 
szulc_serum_2009, semba_low_2009, pilz_vitamin_2009, 
melamed_ml_25-hydroxyvitamin_2008, jia_nutritional_2007}, suggesting that 
the association observed in our study might reflect a general phenomenon rather 
than something specific to RCC prognosis. This would be consistent with our 
observation that any association might be stronger among those patients 
diagnosed with early rather than advanced stage tumours. The lack of 
heterogeneity by cause of death provides additional indirect evidence that the 
association between vitamin D and mortality is unlikely to exist exclusively 
among RCC patients. 

In summary, we found that high 25(OH)D$_3$ at diagnosis of RCC was associated 
with lower risk of death. This association was not restricted to RCC 
cause-specific death, and appeared stronger for early stage disease, 
supporting the notion of 25(OH)D$_3$ being associated with lower overall death 
rates in general, rather than RCC prognosis specifically. 

\section*{Acknowledgements}
\noindent Funding:
The World Cancer Research Fund (UK) supported the biochemical analyses. This work was 
also supported by a grant from the NCI (U01-CA155309), and the Ministry of Health 
Czech Republic -- DRO (MMCI, 00209805). The funding organization had no role in 
design and conduct of the study; collection, management, analysis, and 
interpretation of the data; preparation, review, or approval of the manuscript. The 
work undertaken by DCM that was reported in this paper was done so during the tenure 
of an IARC-Australia Postdoctoral Fellowship from the International Agency for 
Research on Cancer, supported by the Cancer Council Australia. 

%%%%%%%%%%%%%%%%%%%%%%%%%%%%%%%%%%%%%%%%%%%%%%%%%%%%%%%%%%%%%%%%%%%%
%% #5 references
%%%%%%%%%%%%%%%%%%%%%%%%%%%%%%%%%%%%%%%%%%%%%%%%%%%%%%%%%%%%%%%%%%%%
\clearpage
\bibliographystyle{./bibtex/aje}
\bibliography{./bibtex/library_vitd}
\end{document}


\clearpage
\section*{Tables}

%% Table 1: Descriptive table
\input{../analysis/output/o10_table_descriptive.tex}
\clearpage

%% Table 2: table of estimates from relative risk models
\input{../analysis/output/o11_table_riskmodels.tex}
\clearpage

%% Appendix Table 1: Distribution of vit d by Country
\input{../analysis/output/o10_table_d3_dist.tex}
\clearpage


\section*{Figure legends}
\end{document}
