%% Vitamin D and RCC survival
%% draft manuscript

\documentclass[a4paper,11pt]{article}
%\usepackage[numbers,comma,square,sort&compress]{natbib}
\usepackage{cite}
\renewcommand{\citeleft}{(}
\renewcommand{\citeright}{)}
\usepackage{setspace}

\usepackage[left=2cm,right=2cm,top=2cm,bottom=2cm]{geometry}
\usepackage{graphicx}
\usepackage{amsmath}
\usepackage{amssymb}


\usepackage{threeparttable}
\usepackage{color, soul}
\usepackage{booktabs}
\usepackage{multicol}
\usepackage{pdflscape}
\usepackage{hyperref}
\hypersetup{
  colorlinks=true,
  urlcolor=black,
  citecolor=black,
  linkcolor=black,
}

%% use palatino font
\usepackage{mathpazo}
\linespread{1.15}

%% custom captions
\usepackage[labelformat=empty]{caption}

%% indenting and paragraphing
\parskip 10pt
\parindent 25pt

%\renewcommand{\cite}{\citep}

%%%%%%%%%%%%%%%%%%%%%%%%%%%%%%%%%%%%%%%%%%%%%%%%%%%%%%%%%%%%%%%%%%%%
%% #1 end preamble. Begin document proper
%%%%%%%%%%%%%%%%%%%%%%%%%%%%%%%%%%%%%%%%%%%%%%%%%%%%%%%%%%%%%%%%%%%%

\begin{document}
\doublespace
\noindent \textbf{Title:} Circulating 25-Hydroxyvitamin D$_3$ and 
survival after diagnosis with kidney cancer. 

\noindent \textbf{Authors:} David C Muller$^1$, Mattias Johansson$^1$, David 
Zaridze$^2$, Vladimir Janout$^3$, Ivana Holcatova$^4$, Lenka Foretova$^5$, Dana 
Mates$^6$, Paul Brennan$^1$, Ghislaine Scelo$^1$

{\footnotesize 
\noindent
1. International Agency for Research on Cancer (IARC), Lyon, France \newline
2. Russian N.N. Blokhin Cancer Research Centre, Moscow, Russian Federation 
\newline
3. Department of Preventive Medicine, Faculty of Medicine, Palacky University, 
Olomouc, Czech Republic \newline
4. Charles University in Prague, First Faculty of Medicine, Institute of 
Hygiene 
and Epidemiology, Prague, Czech Republic \newline
5. Department of Cancer Epidemiology and Genetics, Masaryk Memorial Cancer 
Institute, Brno, Czech Republic \newline
6. National Institute of Public Health, Bucharest, Romania
}

\textbf{Corresponding author:} \newline




%%%%%%%%%%%%%%%%%%%%%%%%%%%%%%%%%%%%%%%%%%%%%%%%%%%%%%%%%%%%%%%%%%%%
%% #2 abstract
%%%%%%%%%%%%%%%%%%%%%%%%%%%%%%%%%%%%%%%%%%%%%%%%%%%%%%%%%%%%%%%%%%%%
\begin{abstract}
\textbf{Background:} \newline
\textbf{Methods:} \newline
\textbf{Results:} \newline
\textbf{Conclusions:} \newline
\end{abstract}

%%%%%%%%%%%%%%%%%%%%%%%%%%%%%%%%%%%%%%%%%%%%%%%%%%%%%%%%%%%%%%%%%%%%
%% #3 introduction
%%%%%%%%%%%%%%%%%%%%%%%%%%%%%%%%%%%%%%%%%%%%%%%%%%%%%%%%%%%%%%%%%%%%
\clearpage
\section*{Introduction}
More than 300,000 kidney cancer cases are diagnosed yearly worldwide leading to 
approximately 130,000 deaths \cite{ferlay_cancer_2013}. The prognosis is 
strongly dependent on stage at diagnosis, with around 90\% of stage I patients 
alive five years after diagnosis, comparted with only 10\% of stage IV patients 
\cite{CRUK_kidney_2014}.

Little is known on factors influencing survival after diagnosis of kidney 
cancer, apart from tumor stage and grade. We recently investigated circulating 
vitamin D and kidney cancer onset and survival in a prospective epidemiological 
cohort where blood samples were collected at entry in the cohort, an average 
of 7 years before diagnosis [Muller et al., in press]. We reported an inverse 
association between vitamin D concentrations and risk of subsequent kidney 
cancer, and an increased rate of death after kidney cance diagnosis for both 
low and high concentrations of vitamin D. Available data did not allow thorough 
analysis of cause-specific mortality, nor were we able to adjust for stage. This 
observation prompted us to investigate whether vitamin B6 levels at diagnosis of 
kidney cancer are predictive of subsequent survival, independent of stage or 
other prognostic factors.   

%%%%%%%%%%%%%%%%%%%%%%%%%%%%%%%%%%%%%%%%%%%%%%%%%%%%%%%%%%%%%%%%%%%%
%% #3 methods 
%%%%%%%%%%%%%%%%%%%%%%%%%%%%%%%%%%%%%%%%%%%%%%%%%%%%%%%%%%%%%%%%%%%%
\section*{Methods}
\subsection*{The K2 study}
Participants included patients who were above 18 years of age and diagnosed 
with kidney cancer in one of 6 participating centres in Czech Republic (4 
centres), Romania (Bucharest), and Russia (Moscow). After obtaining informed 
consent, participants were given a standardized face-to-face short lifestyle 
questionnaire covering socio-demographic characteristics, anthropometric 
measures, medical history, family history, and tobacco and alcohol use. 
Clinical and pathological data were abstracted from medical charts and 
pathological reports. A majority of participants underwent nephrectomy and the 
tumor was histologically confirmed. Follow-up for outcome (relapse, vital 
status, and cause of death where relevant) was performed every 6 to 12 months 
after diagnosis, using passive follow-up methods where possible (with 
confirmation of vital status through active follow-up methods in case of 
uncertainties) and active follow-up methods when no linkage to databases was 
possible.

\subsection*{Case-cohort sampling}
Among 2330 participants with questionnaire data available and a diagnosis of 
renal cell carcinoma (RCC) according to ICDO-3, we excluded 125 participants 
with no plasma sample available, 1005 participants with no follow-up data at 
the time of this project, 5 participants with inconsistencies in reference 
dates, and 7 participants with no information on stage. From the 1188 
remaining, we randomly selected 500 participants at baseline (the subcohort). 
We also included all participants who died during follow-up that were not 
randomly selected in to the subcohort ($N=93$), as well as 37 stage IV patients 
that had survived and were not randomly selected, hence a total of 
630 participants for laboratory analysis. Of these, one subject ended up being 
classified as spindle cell sarcoma.

\subsection*{Biosample processing and biochemical analysis}
Venous blood was obtained before or at the time of the nephrectomy, prior to 
any treatment. Details of blood collection, storage, and transport have 
been described previously [CITATION]. Liquid chromatography coupled to tandem 
mass spectrometry was used to separately analyse vitamin D as 25(OH)D$_2$ and 
25(OH)D$_3$ \cite{midttun_determination_2011}. 25(OH)D$_2$ was undetectable in 
the majority of samples, so our analyses focus on 25(OH)D$_3$. The lab is DEQAS 
(www.deqas.org) certified.

\subsection*{Statistical analysis}
To adjust for seasonal variation, we modeled the expected log$_2$ 25(OH)D$_3$ 
concentration as a periodic function of day of blood draw using a pair of sine 
and cosine functions. To create season-adjusted categories, we grouped the 
residuals from this model at quartiles of their distribution among the randomly 
selected subcohort. We used Cox proportional hazards models with time since 
diagnosis (recruitment) as the time scale to estimate hazard ratios (HR) and 
95\% confidence intervals (CI) for all-cause mortality by the season-adjusted 
categories of 25(OH)D$_3$. We also modeled 25(OH)D$_3$ continuously using 
restricted cubic splines with knots at its 10th, 33rd, 67th, and 90th 
percentiles, explicitly adjusting for seasonality by including the pair of sine 
and cosine functions. HRs for cause-specific mortality were calculated in a 
competing risks model using the data augmentation method 
\cite{lunn_applying_1995}. To account for the case-cohort design 
\cite{prentice_case-cohort_1986} we used Barlow's method to weight the 
likelihood and computed robust variance estimates \cite{barlow_robust_1994, 
barlow_analysis_1999}. We investigated potential effect modification by fitting 
interactions between season-adjusted log$_2$ 25(OH)D$_3$ and various factors. 
All models included stage, age at recruitment, and sex as covariates, with the 
baseline hazard stratified by country of recruitment. We additionally adjusted 
for body mass index (BMI, kg/m$^2$), smoking status (never, former, current), 
and alcohol drinking status (never, former, current).
All $p$-values are two sided, and were calculated using the Wald test. 
Statistical analyses were performed using Stata 12.1 for Linux (Stata 
Corporation, College Station, Texas, US) and R version 3.1.1 \cite{r_2014}.


%%%%%%%%%%%%%%%%%%%%%%%%%%%%%%%%%%%%%%%%%%%%%%%%%%%%%%%%%%%%%%%%%%%%
%% #4 results
%%%%%%%%%%%%%%%%%%%%%%%%%%%%%%%%%%%%%%%%%%%%%%%%%%%%%%%%%%%%%%%%%%%%
\section*{Results}
Demographic and clinical characteristics of the study sample by vital status at 
the end of follow-up are presented in Table~1. The sample included a higher 
proportion of men (63\%) than women, and were predominantly recruited from the 
Czech Republic (52\%) and Russia (43\%), with only 5\% or participants 
recruited in Romania. Those participants who survived to the end of follow-up 
had a similar age distribution to those who died during follow-up. 518 
of the 630 cases (82\%) were conventional RCC. 50\% of deaths occurred among 
participants with a stage IV tumor, and 15\% of those surviving to the end of 
follow-up had stage IV diagnoses. In contrast, 72 \% of those surviving to the 
end of follow-up were diagnosed with stage I-II disease.

Hazard of death from any cause was strongly and inversely associated with 
circulating concentrations of 25(OH)D$_3$ (Table~2). After adjusting for 
stage, age, and sex, the hazard was 43\% lower among those in the highest 
compared to the lowest group of seasonally-adjusted concentration 
(HR$_{4\text{vs}1}$ 0.57, 95\% CI [0.34, 0.97]). A competing risks analysis 
found no evidence of heterogeneity by cause of death, suggesting that this 
association was not specific to deaths from RCC (Table~2). The HR for 
continuously varying 25(OH)D$_3$ (relative to a concentration of 50 nmol/L) is 
presented in Figure~1. These estimates corroborate those in Table~2, suggesting 
a monotonic inverse association between 25(OH)D$_3$ and hazard of death.

Figure~2 presents HRs for a doubling in seasonally adjusted 25(OH)D$_3$ 
concentration separately by categories of several potential effect modifiers. 
The estimated magnitude of the association was consistent by, sex, stage, 
histology, history of diabetes, smoking status, and alcohol intake status. 
There was some indication that the association might be stronger among those 
diagnosed at age 65 years or older, those with a history of hypertension, those 
with higher BMI, or those diagnosed with stage I or II RCC, but there was 
little statistical evidence of interaction with any of these factors. 

%%%%%%%%%%%%%%%%%%%%%%%%%%%%%%%%%%%%%%%%%%%%%%%%%%%%%%%%%%%%%%%%%%%%
%% #4 discussion
%%%%%%%%%%%%%%%%%%%%%%%%%%%%%%%%%%%%%%%%%%%%%%%%%%%%%%%%%%%%%%%%%%%%
\section*{Discussion}
We investigated whether differences in circulating concentrations of 
25(OH)D$_3$ at the time of diagnosis of RCC were associated with all-cause and 
cause-specific survival. We observed that higher concentrations of 25(OH)D$_3$ 
were associated with a substantially lower risk of death, and that the 
associatin was not restricted to RCC specific death. We also observed an 
indication that the strength of this association might vary by other factors.
Most notably the association appeared somewhat stronger for those with a 
history of hypertension, advanced age at diagosis, or eary stage disease.

\clearpage
\bibliographystyle{./bibtex/aje}
\bibliography{./bibtex/library_vitd}
\end{document}
Previous reports of prospectively measured circulating 25(OH)D and risk of 
kidney cancer have produced conflicting results. In accordance with our 
results, the Copenhagen City Heart Study (CCHS), a cohort of 9791 people 
including 55 incident kidney cancer cases, reported that a 50\% reduction in 
25(OH)D was associated with increased risk (HR 1.34, 95\% CI [1.04, 1.73]) 
\cite{afzal_low_2013}. In contrast, the vitamin D pooling project found no 
association in an analysis of 775 case-control pairs nested within eight 
prospective cohorts \cite{gallicchio_circulating_2010-1}. These discrepant 
results are not readily explicable. One difference between the pooling project 
and the present study is the method of adjustment for season. The pooling 
project used conditional logistic regression models adjusted for season of blood 
draw (summer/winter), with sensitivity analyses adjusted using the residuals 
from a local polynomial regression \cite{gallicchio_circulating_2010}. 
In contrast, we directly modelled seasonality using smooth trigonometric 
functions. Another difference between the studies is that both the pooling 
project and CCHS used a chemiluminescence immunoassay measuring both
25(OH)D$_2$ and 25(OH)D$_3$, whereas in the present study we employed liquid 
chromatography coupled to tandem mass spectrometry to quantify 25(OH)D$_3$ 
specifically. That said, these 
methodological differences would seem unlikely to fully account for the 
discrepant results which remain unexplained. 

There is a large literature investigating circulating vitamin D and all-cause 
mortality in general populations. Consistent with our observation, many studies 
have reported increased risk of death for people with low concentrations 
\cite{schottker_strong_2013, virtanen_association_2011, hutchinson_low_2010, 
szulc_serum_2009, semba_low_2009, pilz_vitamin_2009, 
melamed_ml_25-hydroxyvitamin_2008, jia_nutritional_2007}. This suggests that 
the association observed in our study may not be specific to post RCC survival, 
but rather a reflection of a general phenomenon. Our observation that high 
levels of 25(OH)D$_3$ might be associated with increased risk of death is 
consistent with results from the Uppsala Longitudinal Study of Adult Men, which 
also suggest a U-shaped association \cite{michaelsson_plasma_2010}. Despite 
this accord, further studies are required to investigate the intriguing 
possibility that both low and high concentrations are associated with all-cause 
mortality.

The principal limitation of our study is that 25(OH)D$_3$ was measured using a 
single blood sample drawn in adulthood. While individual vitamin D measurements 
are reasonably reproducible, intra-individual variation may still be important 
\cite{major_variability_2013}. Further, it is possible that this single  
measurement in adulthood does not capture exposure to vitamin D in an 
etiologically relevant period. 

Our study does have several strengths. Importantly, our sample includes 
participants from 10 European countries from differing geographic latitudes, 
and a wide range of 25(OH)D$_3$ concentrations. Biospecimen handling was 
standardised, and quantification of circulating 25(OH)D$_3$ took place in a 
single laboratory, thus minimising systematic inter-laboratory variation. The 
prospective design of our study, in which 25(OH)D$_3$ concentrations were 
assessed using blood collected prior to diagnosis, minimises the chance that 
any differences between cases and controls are caused by existing tumours. 
Further, the availability of detailed information on potential confounders -- 
particularly the inclusion of circulating cotinine as a biomarker of current 
smoking intensity and creatinine as a marker of renal function --  affords 
additional confidence that the observed associations are not due to residual 
confounding. 

In conclusion, we found that low concentrations of 25(OH)D$_3$ were associated 
with increased risk of RCC as well as lower all-cause mortality among RCC 
cases. High concentrations of 25(OH)D$_3$ might also be associated with 
increased risk of all-cause mortality among RCC cases.

\section*{acknowledgements}


\noindent Funding:
The World Cancer Research Fund (UK) supported the biochemical analyses. The 
funding organization had no role in design and conduct of the study; collection, 
management, analysis, and interpretation of the data; preparation, review, or 
approval of the manuscript. The work undertaken by DCM that was reported in this 
paper was done so during the tenure of an IARC-Australia Postdoctoral Fellowship 
from the International Agency for Research on Cancer, supported by the Cancer 
Council Australia. 

%%%%%%%%%%%%%%%%%%%%%%%%%%%%%%%%%%%%%%%%%%%%%%%%%%%%%%%%%%%%%%%%%%%%
%% #5 references
%%%%%%%%%%%%%%%%%%%%%%%%%%%%%%%%%%%%%%%%%%%%%%%%%%%%%%%%%%%%%%%%%%%%
\clearpage
\bibliographystyle{./bibtex/aje}
\bibliography{./bibtex/library_vitd}

\clearpage
\section*{Tables}

%% Table 1: Descriptive table
\input{../analysis/output/o10_table_descriptive.tex}
\clearpage

%% Table 2: table of estimates from relative risk models
\input{../analysis/output/o11_table_riskmodels.tex}
\clearpage

%% Appendix Table 1: Distribution of vit d by Country
\input{../analysis/output/o10_table_d3_dist.tex}
\clearpage


\section*{Figure legends}
\noindent \textbf{Figure 1.} Seasonal variation of 25(OH)D$_3$ 
concentrations. Scattered points show the measured values. The solid line 
represents the predicted geometric mean concentration given day of blood draw, 
which was modeled as a linear combination of sine and cosine functions. See the 
text of the methods section for further details. Estimates and data are from a 
nested European Prospective investigation into Cancer and Nutrition (EPIC) 
study, 
which recruited participants between 1992 and 2000. \newline \newline

\noindent \textbf{Figure 2.} Odds ratio for RCC as a function of 
circulating  concentration of 25(OH)D$_3$, relative to a concentration of 
50~nmol/L. Log-base-2 25(OH)D$_3$ was modeled as a continuous 
covariate. Solid and dashed 
lines represent the maximum likelihood estimates and 95\% 
confidence intervals respectively. The translucent lines are 1000 draws from 
the 
multivariate normal distribution defined by the maximum likelihood estimates 
and 
their variance covariance matrix, and thus give an indication of the posterior 
density for the odds ratio under a uniform prior on the regression 
coefficients. 
The ``rug plot'' under each panel shows the observed distribution of 
25(OH)D$_3$. Estimates and data are from a nested case-control sample within 
the European Prospective investigation into Cancer and Nutrition (EPIC) study, 
which recruited participants between 1992 and 2000. \newline
\noindent Panel A) Estimates adjusted for age at baseline, sex, country, and 
seasonality (sine and cosine functions of day of blood draw).  \newline 
\noindent Panel B) Estimates after additional adjustment for smoking status at 
baseline (never/former/current), circulating cotinine (quartiles defined among 
the controls), alcohol intake at recruitment (g/day), and body mass index 
(kg/m$^2$).  \newline \newline

\noindent \textbf{Figure 3.} Stratified odds ratios (95\% CI) for 
RCC for a doubling in concentration of 25(OH)D$_3$. Estimates are adjusted for 
age at baseline, sex, country, seasonality (sine and cosine functions of day of 
blood draw), smoking status at baseline (never/former/current), circulating 
cotinine (quartiles defined among the controls), alcohol intake at recruitment 
(g/day), and body mass index (kg/m$^2$). Estimates are from a nested 
case-control sample within the European Prospective investigation into Cancer 
and Nutrition (EPIC) study, which recruited participants between 1992 and 2000. 
\newline
$^a$P-value from likelihood ratio test of interaction terms. \newline
Abbreviations: OR, Odds Ratio; CI, Confidence Interval; BMI, Body Mass Index.  
\newline \newline

\noindent \textbf{Figure 4. Post RCC survival. Estimates from a nested 
case-control sample within the European Prospective investigation into Cancer 
and Nutrition (EPIC) study, which recruited participants between 1992 and 
2000.} \newline
\noindent Panel A) Hazard ratio from a Cox model for all cause mortality post 
RCC 
diagnosis as a function of circulating concentration of 25(OH)D$_3$, 
relative to a concentration of 50~nmol/L. 25(OH)D$_3$ was modeled 
using 
restricted cubic splines with knots at the 10th, 33rd, 67th, and 90th 
percentiles of its distribution. The model was adjusted for age at baseline, 
sex, country, and seasonality (sine and cosine functions of day of blood draw), 
smoking status at baseline (never/former/current), circulating cotinine 
(quartiles defined among the controls), alcohol intake at recruitment (g/day), 
and body mass index (kg/m$s^2$). Solid and 
dashed lines represent the maximum likelihood estimates and 95\% confidence 
intervals respectively. The translucent lines are 1000 draws from the 
multivariate normal distribution defined by the maximum likelihood estimates 
and 
their variance covariance matrix, and thus give an indication of the posterior 
density for the hazard ratio under a uniform prior on the regression 
coefficients. The ``rug plot'' shows the observed distribution of 
25(OH)D$_3$. 
\newline 
\noindent Panel B) Survival function post RCC diagnosis evaluated at given 
concentrations 
of 25(OH)D$_3$, derived from a flexible parametric survival model. 
Restricted cubic splines with knots at the 0th, 33rd, 67th, and 100th 
percentiles of the distribution of uncensored survival times were used to model 
the baseline hazard. Like the Cox model used to derive panel A, 
25(OH)D$_3$ was modeled using restricted cubic splines with knots at 
the 10th, 33rd, 67th, and 90th percentiles of its distribution.

\end{document}
